\chapter{Statistical Analysis Plan Templates}

\section{Template Statistical Analysis Plan for Digital Health Application Trial}

\subsection{Study Overview}

\begin{tcolorbox}[title=Study Information]
\begin{tabular}{|p{4cm}|p{10cm}|}
\hline
\textbf{Protocol Title} & [Full Title of the Study] \\
\hline
\textbf{Protocol Number} & [Study Identifier] \\
\hline
\textbf{Primary Objective} & [Primary Research Question] \\
\hline
\textbf{Secondary Objectives} & [List of Secondary Research Questions] \\
\hline
\textbf{Study Design} & [Description of Study Design] \\
\hline
\textbf{Sample Size} & [Planned Number of Participants] \\
\hline
\textbf{Study Population} & [Key Inclusion/Exclusion Criteria] \\
\hline
\textbf{Study Duration} & [Length of Study] \\
\hline
\textbf{Statistical Analysis Plan Version} & [Version Number and Date] \\
\hline
\end{tabular}
\end{tcolorbox}

\subsection{Randomization and Blinding}

\begin{itemize}
    \item \textbf{Randomization Method}: [Describe the randomization procedure, e.g., stratified block randomization]
    
    \item \textbf{Allocation Ratio}: [Specify the ratio of participants in each arm, e.g., 1:1]
    
    \item \textbf{Stratification Factors}: [List any stratification variables used in randomization]
    
    \item \textbf{Blinding Procedures}: [Describe which study personnel are blinded to treatment assignment]
    
    \item \textbf{Unblinding Procedures}: [Describe circumstances under which unblinding may occur]
\end{itemize}

\subsection{Analysis Populations}

\begin{itemize}
    \item \textbf{Intent-to-Treat (ITT) Population}: All randomized participants, analyzed according to assigned treatment group regardless of protocol adherence.
    
    \item \textbf{Modified Intent-to-Treat (mITT) Population}: [Define specific criteria, e.g., all randomized participants who complete at least one post-baseline assessment]
    
    \item \textbf{Per-Protocol (PP) Population}: All participants who complete the study without major protocol deviations and with minimum specified application usage.
    
    \item \textbf{Safety Population}: All participants who receive the digital intervention and have at least one post-baseline safety assessment.
\end{itemize}

\subsection{Handling of Missing Data}

\begin{itemize}
    \item \textbf{Primary Approach}: [Specify primary method, e.g., mixed models for repeated measures]
    
    \item \textbf{Missing Data Patterns}: Patterns of missing data will be described and classified as Missing Completely at Random (MCAR), Missing at Random (MAR), or Missing Not at Random (MNAR).
    
    \item \textbf{Imputation Method}: [Describe imputation approach if applicable, e.g., multiple imputation]
    
    \item \textbf{Sensitivity Analyses}: [List planned sensitivity analyses to evaluate the robustness of results to missing data assumptions]
\end{itemize}

\subsection{Analysis of Primary Outcome}

\begin{tcolorbox}[title=Primary Outcome Analysis]
\begin{tabular}{|p{4cm}|p{10cm}|}
\hline
\textbf{Primary Outcome} & [Name and definition of primary outcome measure] \\
\hline
\textbf{Measurement Time Points} & [When outcome is assessed] \\
\hline
\textbf{Analysis Population} & [Usually Intent-to-Treat] \\
\hline
\textbf{Statistical Method} & [Specific test or model to be used] \\
\hline
\textbf{Covariates} & [Any adjustment variables to be included] \\
\hline
\textbf{Handling of Missing Data} & [Approach specific to primary outcome] \\
\hline
\textbf{Sensitivity Analyses} & [Alternative approaches to test robustness] \\
\hline
\end{tabular}
\end{tcolorbox}

\subsection{Analysis of Secondary Outcomes}

For each secondary outcome, specify:

\begin{itemize}
    \item \textbf{Outcome Definition}: [Clear definition of the outcome measure]
    
    \item \textbf{Analysis Population}: [Population for this specific analysis]
    
    \item \textbf{Statistical Method}: [Test or model to be applied]
    
    \item \textbf{Multiplicity Adjustment}: [Method for controlling Type I error across multiple tests]
\end{itemize}

\subsection{Digital Health-Specific Analyses}

\begin{itemize}
    \item \textbf{Engagement Metrics Analysis}:
    \begin{itemize}
        \item Frequency of use (sessions per week)
        \item Duration of use (minutes per session)
        \item Feature utilization (proportion of available features used)
        \item Engagement decay (change in usage patterns over time)
    \end{itemize}
    
    \item \textbf{Dose-Response Analysis}:
    \begin{itemize}
        \item Definition of digital dose (e.g., total minutes of active use)
        \item Statistical approach for relating dose to outcomes
        \item Threshold analysis to identify minimum effective dose
    \end{itemize}
    
    \item \textbf{Usage Pattern Analysis}:
    \begin{itemize}
        \item Clustering methods to identify usage typologies
        \item Time-of-day usage patterns
        \item Sequential pattern analysis of feature use
    \end{itemize}
\end{itemize}

\subsection{Subgroup Analyses}

\begin{itemize}
    \item \textbf{Prespecified Subgroups}:
    \begin{itemize}
        \item Cognitive status (e.g., MCI vs. mild dementia)
        \item Age groups
        \item Technology experience levels
        \item Presence/absence of caregiver support
    \end{itemize}
    
    \item \textbf{Statistical Approach}: [Method for testing treatment effect heterogeneity, e.g., interaction terms in regression models]
    
    \item \textbf{Interpretation Guidance}: Subgroup analyses will be considered exploratory and hypothesis-generating rather than confirmatory.
\end{itemize}

\subsection{Safety Analysis}

\begin{itemize}
    \item \textbf{Adverse Event Tabulation}: Frequencies and percentages of adverse events by type, severity, and relatedness to the digital intervention.
    
    \item \textbf{Technology-Specific Safety Outcomes}: Analysis of technology-related adverse events such as privacy breaches, software malfunctions, or digital fatigue.
    
    \item \textbf{Psychological Safety Measures}: Analysis of measures related to distress, frustration, or other negative psychological responses to the intervention.
\end{itemize}

\subsection{Interim Analyses}

\begin{itemize}
    \item \textbf{Timing}: [When interim analyses will be conducted]
    
    \item \textbf{Purpose}: [Objectives of interim analyses, e.g., safety monitoring, futility assessment]
    
    \item \textbf{Statistical Methods}: [Approaches that account for multiple looks at the data]
    
    \item \textbf{Stopping Rules}: [Criteria that would lead to early termination]
\end{itemize}

\section{Example Statistical Analysis Code Templates}

\subsection{R Code Template for Mixed Effects Model}

\begin{tcolorbox}[title=R Code for Mixed Effects Model]
\begin{verbatim}
# Load required packages
library(lme4)
library(lmerTest)
library(ggplot2)

# Import and prepare data
data <- read.csv("study_data.csv")

# Define analysis population
itt_population <- data[data$randomized == 1, ]

# Create factor variables
itt_population$treatment <- factor(itt_population$treatment, 
                                  levels = c(0, 1), 
                                  labels = c("Control", "Intervention"))
itt_population$time <- factor(itt_population$visit)

# Mixed effects model for repeated measures
model <- lmer(outcome ~ treatment * time + baseline_score + 
              age + gender + (1|subject_id), 
              data = itt_population)

# Summary of model results
summary(model)

# Extract treatment effect at final time point
final_time <- max(as.numeric(itt_population$time))
contrasts_result <- emmeans(model, ~ treatment | time)
final_contrast <- contrast(contrasts_result[final_time], method = "pairwise")
print(final_contrast)

# Visualization of results
ggplot(itt_population, aes(x = time, y = outcome, group = treatment, color = treatment)) +
  stat_summary(fun = mean, geom = "point", size = 3) +
  stat_summary(fun = mean, geom = "line") +
  stat_summary(fun.data = mean_se, geom = "errorbar", width = 0.2) +
  labs(title = "Change in Outcome Over Time by Treatment Group",
       x = "Visit", y = "Outcome Score") +
  theme_minimal()
\end{verbatim}
\end{tcolorbox}

\subsection{R Code Template for Engagement Analysis}

\begin{tcolorbox}[title=R Code for Engagement Analysis]
\begin{verbatim}
# Load required packages
library(dplyr)
library(ggplot2)
library(cluster)

# Import usage data
usage_data <- read.csv("usage_logs.csv")

# Calculate engagement metrics
engagement <- usage_data %>%
  group_by(subject_id) %>%
  summarize(
    total_sessions = n(),
    avg_duration = mean(session_duration, na.rm = TRUE),
    total_minutes = sum(session_duration, na.rm = TRUE),
    days_used = n_distinct(as.Date(session_start)),
    features_used = n_distinct(feature),
    feature_diversity = n_distinct(feature) / n(),
    longest_gap = max(diff(as.Date(session_start))),
    pct_completed = mean(completed_session) * 100
  )

# Merge with outcome data
merged_data <- inner_join(engagement, outcome_data, by = "subject_id")

# Dose-response analysis
dose_model <- lm(final_outcome ~ total_minutes + baseline_score + age + gender,
                data = merged_data)
summary(dose_model)

# Non-linear dose effects
dose_model_nl <- gam(final_outcome ~ s(total_minutes) + baseline_score + age + gender,
                    data = merged_data)
summary(dose_model_nl)
plot(dose_model_nl)

# Cluster analysis to identify usage patterns
usage_matrix <- usage_data %>%
  group_by(subject_id, feature) %>%
  summarize(time_spent = sum(duration)) %>%
  pivot_wider(names_from = feature, values_from = time_spent, values_fill = 0) %>%
  select(-subject_id) %>%
  scale()

# Determine optimal number of clusters
wss <- sapply(1:10, function(k) {
  kmeans(usage_matrix, centers = k, nstart = 25)$tot.withinss
})
plot(1:10, wss, type = "b", xlab = "Number of Clusters", ylab = "Within-cluster Sum of Squares")

# K-means clustering
km <- kmeans(usage_matrix, centers = 3, nstart = 25)
usage_clusters <- data.frame(subject_id = unique(usage_data$subject_id),
                            cluster = km$cluster)

# Analyze outcomes by engagement cluster
cluster_analysis <- merged_data %>%
  inner_join(usage_clusters, by = "subject_id") %>%
  group_by(cluster) %>%
  summarize(
    n = n(),
    mean_outcome = mean(final_outcome, na.rm = TRUE),
    sd_outcome = sd(final_outcome, na.rm = TRUE)
  )
print(cluster_analysis)
\end{verbatim}
\end{tcolorbox}

\subsection{STATA Code Template for Multiple Imputation}

\begin{tcolorbox}[title=STATA Code for Multiple Imputation]
\begin{verbatim}
// Import data
import delimited "study_data.csv", clear

// Examine missing data patterns
misstable patterns outcome_baseline outcome_week4 outcome_week8 outcome_week12

// Multiple imputation
mi set wide
mi register imputed outcome_week4 outcome_week8 outcome_week12
mi register regular subject_id treatment age gender outcome_baseline

// Imputation model
mi impute chained ///
    (regress) outcome_week4 outcome_week8 outcome_week12 = ///
    treatment age gender outcome_baseline, ///
    add(20) rseed(12345)

// Analysis of imputed data
mi estimate, cmdok: mixed outcome_week12 treatment outcome_baseline age gender || site:

// Sensitivity analysis with complete cases only
preserve
drop if missing(outcome_week12)
mixed outcome_week12 treatment outcome_baseline age gender || site:
restore

// Visualize imputation results
mi xeq 1/5: twoway (scatter outcome_week12 outcome_baseline), ///
    subtitle("Imputation #`_mi_m'")
\end{verbatim}
\end{tcolorbox}

\section{Sample Size and Power Calculation Templates}

\subsection{Power Calculation for Continuous Outcomes}

\begin{tcolorbox}[title=Sample Size Calculation for Continuous Primary Outcome]
For a two-arm randomized trial with a continuous primary outcome, the required sample size per group can be calculated as:

\begin{equation}
n = \frac{2\sigma^2(Z_{\alpha/2} + Z_{\beta})^2}{\Delta^2}
\end{equation}

Where:
\begin{itemize}
    \item $\sigma^2$ is the variance of the outcome (estimated as [value])
    \item $Z_{\alpha/2}$ is the critical value for a two-sided significance level $\alpha$ (1.96 for $\alpha = 0.05$)
    \item $Z_{\beta}$ is the critical value for power $1-\beta$ (0.84 for 80\% power)
    \item $\Delta$ is the minimum clinically important difference (MCID) to detect (estimated as [value])
\end{itemize}

For this study:
\begin{itemize}
    \item Estimated standard deviation: [value]
    \item MCID: [value]
    \item Desired power: 80\%
    \item Two-sided significance level: 5\%
    \item Calculated sample size per group: [calculated value]
    \item Adjusted for expected attrition (20\%): [adjusted value]
\end{itemize}

The study will therefore aim to recruit a total of [total sample size] participants.
\end{tcolorbox}

\subsection{Power Calculation for Time-to-Event Outcomes}

\begin{tcolorbox}[title=Sample Size Calculation for Time-to-Event Outcome]
For a study examining time to a specific event (e.g., time to application abandonment), the required sample size can be calculated as:

\begin{equation}
n = \frac{4 (Z_{\alpha/2} + Z_{\beta})^2}{[\ln(HR)]^2 \times p_{event}}
\end{equation}

Where:
\begin{itemize}
    \item $HR$ is the hazard ratio to detect (estimated as [value])
    \item $p_{event}$ is the proportion of participants expected to experience the event (estimated as [value])
    \item $Z_{\alpha/2}$ and $Z_{\beta}$ are as defined previously
\end{itemize}

For this study:
\begin{itemize}
    \item Expected hazard ratio: [value]
    \item Expected event rate: [value]
    \item Desired power: 80\%
    \item Two-sided significance level: 5\%
    \item Calculated total sample size: [calculated value]
    \item Adjusted for expected attrition (20\%): [adjusted value]
\end{itemize}
\end{tcolorbox}

\subsection{Power Calculation for Cluster-Randomized Trials}

\begin{tcolorbox}[title=Sample Size Calculation for Cluster-Randomized Trial]
For a cluster-randomized trial (e.g., randomizing care facilities rather than individual participants), the required sample size must be inflated by the design effect:

\begin{equation}
DE = 1 + (m - 1) \times ICC
\end{equation}

Where:
\begin{itemize}
    \item $DE$ is the design effect
    \item $m$ is the average cluster size (estimated as [value])
    \item $ICC$ is the intracluster correlation coefficient (estimated as [value])
\end{itemize}

The total required sample size is then:

\begin{equation}
n_{cluster} = n_{individual} \times DE
\end{equation}

For this study:
\begin{itemize}
    \item Sample size for individual randomization: [value] (as calculated above)
    \item Average cluster size: [value]
    \item Estimated ICC: [value]
    \item Design effect: [calculated value]
    \item Required sample size accounting for clustering: [adjusted value]
    \item Number of clusters needed (adjusted for 20\% attrition): [final value]
\end{itemize}
\end{tcolorbox}